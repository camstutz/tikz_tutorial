%% LaTeX-Beamer template for KIT design
%% by Erik Burger, Christian Hammer
%% title picture by Klaus Krogmann
%%
%% version 2.1
%%
%% mostly compatible to KIT corporate design v2.0
%% http://intranet.kit.edu/gestaltungsrichtlinien.php
%%
%% Problems, bugs and comments to
%% burger@kit.edu

\documentclass[18pt]{beamer}

%% SLIDE FORMAT

% use 'beamerthemekit' for standard 4:3 ratio
% for widescreen slides (16:9), use 'beamerthemekitwide'

\usepackage{templates/beamerthemekit}
% \usepackage{templates/beamerthemekitwide}

%% TITLE PICTURE

% if a custom picture is to be used on the title page, copy it into the 'logos'
% directory, in the line below, replace 'mypicture' with the 
% filename (without extension) and uncomment the following line
% (picture proportions: 63 : 20 for standard, 169 : 40 for wide
% *.eps format if you use latex+dvips+ps2pdf, 
% *.jpg/*.png/*.pdf if you use pdflatex)

\titleimage{KIT-Titel}

%% TITLE LOGO

% for a custom logo on the front page, copy your file into the 'logos'
% directory, insert the filename in the line below and uncomment it

\titlelogo{itiv-logo}

% (*.eps format if you use latex+dvips+ps2pdf,
% *.jpg/*.png/*.pdf if you use pdflatex)

%% TikZ INTEGRATION

% use these packages for PCM symbols and UML classes
% \usepackage{templates/tikzkit}
% \usepackage{templates/tikzuml}

% the presentation starts here

\title[Short title]{TikZ Tutorial}
\subtitle{KSETA Doktorandenworkshop 2014}
\author{Christian Amstutz, Tanja Harbaum, Ewa Holt}

\institute{}

% Bibliography

\usepackage[citestyle=authoryear,bibstyle=numeric,hyperref,backend=biber]{biblatex}
\addbibresource{templates/example.bib}
\bibhang1em

%*****************************************
% our packages
%*****************************************

\usepackage{listings}

\usepackage{tikz}
\usetikzlibrary{positioning}
\usetikzlibrary{arrows}

%*****************************************
% our definitions
%*****************************************

\lstdefinestyle{tikzstyle}{
  language=[LaTeX]tex,
  basicstyle=\footnotesize\ttfamily,
  keywordstyle=\bfseries\color{green!40!black},
  commentstyle=\itshape\color{purple!40!black},
  identifierstyle=\color{blue},
  stringstyle=\color{orange},
}

\lstnewenvironment{tikzcode}%
{\lstset{style=tikzstyle}}%
{}

\newcommand{\tikzcodefile}[1]{\lstinputlisting[style=tikzstyle]{#1}}


%*****************************************
%*****************************************
% the presentation
%*****************************************
%*****************************************


\begin{document}

% change the following line to "ngerman" for German style date and logos
\selectlanguage{english}


%*****************************************
%title page
%*****************************************

\begin{frame}
\titlepage
\end{frame}

%*****************************************
%table of contents
%*****************************************

%\begin{frame}{Outline/Gliederung}
%\tableofcontents
%\end{frame}

%*****************************************

\begin{frame}[fragile]{Setting up the Environment in \LaTeX}

\begin{tikzcode}
\documentclass{standalone}
\usepackage{tikz}
\begin{document}
\begin{tikzpicture}
  % TikZ commands go here  
\end{tikzpicture}
\end{document}
\end{tikzcode}

\end{frame}

%*****************************************

\begin{frame}{Scaling Effects}

\begin{tikzpicture}
  \node[fill=kit-green100,minimum width=2cm] (KIT) at (0,0) {KIT};
  \node[fill=kit-green100,minimum width=2cm] (KSETA) at (0,-2) {KSETA}
    edge[<-] (KIT);
\end{tikzpicture}

\begin{tikzpicture}[scale=1.5]
  \node[fill=kit-green100,minimum width=2cm] (KIT) at (0,0) {KIT};
  \node[fill=kit-green100,minimum width=2cm] (KSETA) at (0,-2) {KSETA}
    edge[<-] (KIT);
\end{tikzpicture}

\end{frame}

%*****************************************

\begin{frame}{Exercise 3: Plot}

\note{
  - Using plots from different tools, these would look differently
  - Formulas in plots
  - Line width adaptation
  - Scaling effects
  - nicer view
  
  more information: Chapter 19 of the official documentation
}

\begin{tikzpicture}[domain=0.2:6]
  
  \draw[->, >=stealth'] (-0.2,0) -- (7,0) node[right] {$x$};
  \draw[->, >=stealth'] (0,-0.2) -- (0,6) node[above] {$f(x)$};
  
  \foreach \x in {0.5,1,1.5,2,2.5,3,3.5,4,4.5,5,5.5,6,6.5}
    \draw (\x,2pt) -- (\x,-3pt);
  \foreach \x in {0,1,2,3,4,5,6}
    \node at (\x,-6pt) [anchor=north] {\footnotesize $\x$};    
  \foreach \y/\ytext in {0.5,1,1.5,2,2.5,3,3.5,4,4.5,5,5.5}
    \draw (2pt,\y) -- (-3pt,\y cm);
  \foreach \y/\ytext in {0,1,2,3,4,5}
    \node at (-6pt,\y) [anchor=east] {\footnotesize $\ytext$};
  
  \draw plot[only marks, mark=x, mark options={kit-blue100, thick}]
      file {working_material/measurement.dat};
  \draw[color=kit-green100] plot[smooth] (\x, {1+pow((1/3)*\x, 2)})
      node[right, xshift=6mm] {$f(x) = 1+\frac{1}{3}x^{2}$};

\end{tikzpicture}


\end{frame}

%*****************************************

\begin{frame}[fragile]{Exercise 3: Plot - Solution}

\tikzcodefile{tikz_bilder/exercise_plot.tex}

\end{frame}

%*****************************************

\begin{frame}{More information}

\end{frame}

%*****************************************

\begin{frame}

\vspace{1cm}  
\begin{center}
  \begin{Huge}Thank you for your attention\end{Huge}
\end{center}
  
\end{frame}

%*****************************************

\begin{frame}

\end{frame}

%*****************************************
\section{Section 1}
\subsection{Subsection 1.1}
\begin{frame}{Example slide A}
\begin{itemize}
\item PCM, Citation: \cite{becker2008a} %\language
\pause
\item Bullet point 2
\item \dots
\end{itemize}
\end{frame}

%*****************************************
\subsection{Subsection 1.2}
\begin{frame}{Example slide B}
\begin{block}{Block 1}
\begin{itemize}
\item Bullet point 1
\pause
\item Bullet point 2
\item \dots
\end{itemize}
\end{block}
\end{frame}

%*****************************************
\section{Section 2}
\begin{frame}{Example slide C}
\begin{exampleblock}{Example 1}
\begin{itemize}
\item Bullet point 1
\pause
\item Bullet point 2
\item \dots
\end{itemize}
\end{exampleblock}
\end{frame}

%*****************************************
\appendix
\beginbackup

\begin{frame}[allowframebreaks]{References}
\printbibliography
\end{frame}

\backupend

\end{document}
