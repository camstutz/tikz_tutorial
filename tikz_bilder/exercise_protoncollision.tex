\begin{tikzpicture}[scale=0.7, transform shape]
\tikzset{proton/.style={circle, black, thick, fill=red, 
		minimum width=1.5cm,minimum height=1.5cm, draw}}
\tikzset{neutron/.style={circle, black, thick, fill=gray, 
		minimum width=1.5cm,minimum height=1.5cm, draw}}
\tikzset{collision/.style={star, star points=8, 
		star point ratio=0.2, black, thick, fill=yellow, 
		minimum width=0.5cm,minimum height=0.5cm, draw}}
\tikzset{neutrino/.style={circle, black, thick, fill=blue, 
		minimum width=0.8cm,minimum height=0.8cm, draw}}
\tikzset{positron/.style={circle, black, thick, fill=yellow, 
		minimum width=1.2cm,minimum height=1.2cm, draw}}
\tikzset{myarrow/.style={-latex, shorten >=0.5cm, shorten <=0.5cm, 
		very thick}}

\node[proton] (proton1) {};
\node[font=\Huge] {\textbf{p}};
\node[proton, below = 5cm] (proton2) {};
\node[font=\Huge] at (proton2) {\textbf{p}};
\node[collision,  below right = 2.125cm and 4cm of proton1] 
		(collision1) {};
\node[positron, right = 8cm of proton1] (positron1) {};
\node[font=\Huge] at (positron1) {\textbf{e$^+$}};
\node[neutrino, right = 8cm of proton2] (neutrino1) {};
\node[font=\Huge] at (neutrino1) {\textbf{$\nu$}};
\node[proton, below right = 1.25cm and 10cm of proton1] 
		(proton3) {};
\node[font=\Huge] at (proton3) {\textbf{p}};
\node[neutron, below of = proton3] (neutron1) {};
\node[font=\Huge] at (neutron1) {\textbf{n}};

\draw[myarrow] (proton1) -- (collision1);
\draw[myarrow] (proton2) -- (collision1);
\draw[myarrow] (collision1) -- (positron1);
\draw[myarrow] (collision1) -- (neutrino1);
\draw[myarrow] (collision1) -- (proton3.south west);
\end{tikzpicture}